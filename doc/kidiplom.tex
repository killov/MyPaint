%%%  Ukázkový text a dokumentace stylu pro text závěrečné (bakalářské a
%%%  diplomové) práce na KI PřF UP v Olomouci
%%%  Copyright (C) 2012 Martin Rotter, <rotter.martinos@gmail.com>
%%%  Copyright (C) 2014 Jan Outrata, <jan.outrata@upol.cz>


%%  Pro získání PDF souboru dokumentu je třeba tento zdrojový text v
%%  LaTeXu přeložit (dvakrát) programem pdfLaTeX.

%%  V případě použití programu BibLaTeX pro tvorbu seznamu literatury
%%  je poté ještě třeba spustit program Biber s parametrem jméno
%%  souboru zdrojového textu bez přípony a následně opět (dvakrát)
%%  přeložit zdrojový text programem pdfLaTeX.

%%  Postup získání Postscriptového souboru je popsán v dokumentaci.


%%  Třída dokumentu implementující styl pro závěrečnou práci. Vybrané
%%  nepovinné parametry (ostatní v dokumentaci):

%%  'master' pro sazbu diplomové práce, jinak se sází bakalářská práce

%%  'field=kód' pro Váš studijní obor, kódy pro diplomovou práci 'uvt'
%%  pro Učitelství výpočetní techniky pro střední školy a 'binf' pro
%%  Bioinformatiku, jinak je výchozí Informatika, a pro bakalářskou
%%  práci 'ainfk' pro Aplikovanou informatiku v kombinované formě,
%%  'inf' pro Informatiku, 'infv' pro Informatiku pro vzdělávání a
%%  'binf' pro Bioinfomatiku, jinak je výchozí Aplikovaná informatika
%%  v prezenční formě

%%  'printversion' pro sazbu verze pro tisk (nebarevné logo a odkazy,
%%  odkazy s uvedením adresy za odkazem, ne odkazy do rejstříku),
%%  jinak verze pro prohlížeč

%%  'biblatex' pro zapnutí podpory pro sazbu bibliografie pomocí
%%  BibLaTeXu, jinak je výchozí sazba v prostředí thebibliography

%%  slovak pro slovenský, jinak je výchozí czech pro český

%%  'font=sans' pro bezpatkový font (Iwona Light), jinak výchozí
%%  patkový (Latin Modern)

\documentclass[
%  master,
  field=inf,
%  printversion,
  biblatex,
%  language=english,
%  font=sans,
  glossaries,
  index
]{kidiplom}

%% Informace pro úvodní strany. V jazyku práce (pokud není v komentáři
%% uvedeno česky) a anglicky. Uveďte všechny, u kterých není v
%% komentáři uvedeno, že jsou volitelné. Při neuvedení se použijí
%% výchozí texty. Text pro jiný než nastavený jazyk práce (nepovinným
%% parametrem language makra \documentclass, výchozí český) se zadává
%% použitím makra s uvedením jazyka jako nepovinného parametru.

%% Název práce, česky a anglicky. Měl by se vysázet na jeden řádek.
\title{Grafický editor s výstupem do HTML}
\title[english]{Graphic editor with export to HTML}

%% Volitelný podnázev práce, česky a anglicky. Měl by se vysázet na
%% jeden řádek. Výchozí je prázdný.
\subtitle{}
\subtitle[english]{}

%% Jméno autora práce. Makro nemá nepovinný parametr pro uvedení
%% jazyka.
\author{Zdeněk Mazurák}

%% Jméno vedoucího práce (včetně titulů). Makro nemá nepovinný
%% parametr pro uvedení jazyka.
\supervisor{RNDr. Arnošt Večerka}

%% Volitelný rok odevzdání práce. Výchozí je aktuální (kalendářní)
%% rok. Makro nemá nepovinný parametr pro uvedení jazyka.
%\yearofsubmit{\the\year}

%% Anotace práce, včetně anglické (obvykle překlad z jazyka
%% práce). Jeden odstavec!
\annotation{Cílem této bakalářské práce je vytvoření grafického editoru pro Windows.
Aplikace umožní uživateli kreslit a editovat grafické tvary a text. Všem objektům bude možno nastavit výplň či okraj barvou či gradientem. Editor bude umožňovat export do formátu HTML.}

\annotation[english]{The aim of this bachelor's thesis is to create the graphic editor for Windows. The application will allow the user to draw and edit graphical shapes and text. The user will be able to change the color or gradient of both filling and frame of all present subjects. The editor will allow an exportation to HTML format.}

%% Klíčová slova práce, včetně anglických. Oddělená (obvykle) středníkem.
\keywords{grafický editor; grafické tvary; HTML CANVAS}
\keywords[english]{graphic editor; graphic shapes; HTML CANVAS}

%% Volitelná specifikace příloh textu práce, i anglicky. Výchozí je '1
%% CD/DVD'.
%\supplements{jedno kulaté placaté CD/DVD s malou kulatou dírou uprostřed}
%\supplements[english]{one round flat CD/DVD with a small round hole in the middle}

%% Volitelné poděkování. Stručné! Výchozí je prázdné. Makro nemá
%% nepovinný parametr pro uvedení jazyka.
%%\thanks{Děkuji, děkuji, děkuji.}

%% Cesta k souboru s bibliografií pro její sazbu pomocí BibLaTeXu
%% (zvolenou nepovinným parametrem biblatex makra
%% \documentclass). Použijte pouze při této sazbě, ne při (výchozí)
%% sazbě v prostředí thebibliography.
\bibliography{bibliografie.bib}

%% Další dodatečné styly (balíky) potřebné pro sazbu vlastního textu
%% práce.
\usepackage{lipsum}

\begin{document}
%% Sazba úvodních stran -- titulní, s bibliografickými údaji, s
%% anotací a klíčovými slovy, s poděkováním a prohlášením, s obsahem a
%% se seznamy obrázků, tabulek, vět a zdrojových kódů (pokud jejich
%% sazba není vypnutá).
\maketitle

%% Vlastní text závěrečné práce. Pro povinné závěry, před přílohami,
%% použijte prostředí kiconclusions. Povinná je i příloha s obsahem
%% přiloženého CD/DVD.

%% -------------------------------------------------------------------

\newcommand{\BibLaTeX}{\textsc{Bib}\LaTeX}

\section{Úvod}
Při tvorbě aplikace byl kladen důraz na jednoduché a uživatelsky přívětivé rozhraní. Hlavní inspirace pochází ze známého programu Malování z MS Windows, který je jednoduchý, ale má řadu nevýhod, například nepodporuje průhlednost a hlavně všechno co je jednou nakresleno, už později nelze změnit. Oproti malování náš grafický editor tyto výhody nemá. Obsahuje navíc správu vrstev, u grafických objektů kromě barev lze použít i přechody barev (gradienty) a máme zde lineární a radiální.

Program není určený pro profesionály, ale pro lidi, kteří si v normálním případě vystačí pouze s Malováním z MS Windows, ale občas se jim hodí nějaké ty pokročilé funkce. Pro většinu práce s obrázky si vystačíme pouze se základními funkcemi, proto není důležité mít přeplácaný program plný zbytečných funkcí, které nám budou brát na přehlednosti.

Pro obyčejné smrtelníky ve světě grafiky je důležité, aby uživatelské rozhraní bylo jednoduché a přehledné. Toho docílíme, že budeme vycházet z jednoduchých principů kreslení.

\section{Uživatelské rozhraní}
Následující kapitola popisuje rozložení prvků v uživatelském rozhraní a jejich funkce.

\subsection{Horní panel}
Horní panel aplikace slouží pro výběr nástroje pro práci. Funguje jako přepínačm současne lze zvolit pouze jeden nástroj. Následně uvedu jednotlivé nástroje.

\begin{itemize}
\item Nástroj pro výběr - slouží pro výběr části obrázku, který lze následně klávesovou zkratkou Ctrl+C zkopírovat do schránky a vložit výřez do jiného programu, nebo jej naklonovat.
\item Nástroj pro výběr grafických objektů - slouží pro označení objektu, který chceme editovat.
\item Úsečka - pro kreslení obyčejné úsečky mezi dvěma body.
\item 
\end{itemize}

\section{Styly pro psaní bakalářských a diplomových prací}
Toto jsou styly pro psaní bakalářských a diplomových prací přes typografický systém \LaTeX{}, tedy \textbf{kistyles}.

\subsection{Požadavky a podprovaná prostředí}
Sada balíku \textbf{kistyles} podporuje následující distribuce systému \LaTeX{}:
\begin{itemize}
\item \TeX{} Live.
\end{itemize}

Jsou podporovány všechny výstupní ovladače, tedy jak \textbf{dvi}, tak \textbf{pdf} i \textbf{ps}. Funkčnost zmiňovaných distribucí byla ověřena na několika operačních systémech, mezi které patří:
\begin{enumerate}
\item Windows $8.1$,
\item Archlinux,
\item Debian.
\end{enumerate}

Důrazně se doporučuje používat aktuální verzi dané distribuce systému \LaTeX{}.

%%%  Po přeložení programem CSLaTeX (třikrát) je potřeba použít
%%%  program DVIPS a takto získaný PostScriptový soubor vytisknout
%%%  na PostScriptové tiskárně nebo pomocí programu GhostScript.
%%%
%%%  Rovněž je možné použít program DVIPDFM a vytvořit z dokumentu
%%%  soubor ve formátu PDF včetně hypertextových odkazů.

\subsection{Přepínače}
Styl kidiplom je z hlediska uživatele zastoupen ekvivalentně nazvanou třídou, kterou je třeba volat na záčátku dokumentu:
\begin{kicode}{TeX}{}{Volání třídy \textbf{kidiplom}}
\documentclass[
  master=true,
  font=sans,
  printversion=false,
  joinlists=true,
  glossaries=true,
  figures=true,
  tables=true,
  sourcecodes=true,
  theorems=true,
  bibencoding=utf8,
  language=czech,
  encoding=utf8,
  field=inf,
  index=true,
  biblatex=true
]{kidiplom}
\end{kicode}

Následuje přehled přepínačů, je vždy uvedeno jméno přepínač, včetně výchozí hodnoty. Přepínače uvádí tabulka \ref{tab:prepinace}.

\begin{table}
\begin{center}
\caption{Seznam přepínačů}\label{tab:prepinace}
\scalebox{0.95}{\begin{tabular}{>{\bfseries}l >{\ttfamily}c L{8cm}}
{\normalfont Přepínač} & {\normalfont Výchozí hodnota} & {\normalfont Popis} \\
\hline
master & false & Povolí nebo zakáže režim diplomové práce. Výchozí režim je tedy bakalářská práce. \\

field & ainfp & Specifikuje studijní obor:\newline
\begin{description}
\item[ainf] Aplikovaná informatika\,--\,prezenční,
\item[ainfk] Aplikovaná informatika\,--\,kombinovaná,
\item[inf] Informatika\,--\,prezenční,
\item[infv] Informatika ve vzdělávání\,--\,kombinovaná,
\item[binf] Bioinformatika\,--\,prezenční.
\end{description} \\

font & serif & Zapne či vypne podporu pěkného bezpatkového fontu. Možné hodnoty jsou:\newline
\begin{description}
\item[sans] Bezpatkové písmo (písmo Iwona).
\item[serif] Patkové písmo (písmo Computer Modern).
\end{description} \\

%%  'encoding=kódování' pro kódování tohoto a vložených zdrojových
%%  textů v kódování jiném než výchozím utf8
encoding & utf8 & Kódování souboru dokumentu, doporučuje se ponechat výchozí hodnotu. \\

bibencoding & utf8 & Kódování souboru bibliografie. Tato volba má smysl pouze, pokud je použita bibliografie skrze balíček \BibLaTeX{}. \\

language & czech & Jazyk práce. \\

printversion & false & Je-li zapnuto, pak budou odkazy vysázeny optimalizovaně pro knižní sazbu. Tuto volbu je nutno použít pro tisk práce. \\

%%% Nepovinné argumenty `tables' a `figures' použijte pouze v případě,
%%% že váš dokument obsahuje tabulky a obrázky a chcete vytvořit
%%% jejich seznamy za obsahem.
%%%
%%% Argument `joinlists' způsobí zřetězení obsahu a seznamů tabulek a obrázků.
%%% Není-li použít, všechny seznamy jsou uvedeny na samostatných stránkách.

joinlists & true & Je-li zapnuto, pak seznamy obrázků, tabulek, vět a
zdrojových kódů sázené za obsahem nebudou rozděleny na samostatné stránky. \\

figures & true & Je-li zapnuto, pak v seznamech položek bude zahrnut seznam obrázků. \\

tables & true & Je-li zapnuto, pak v seznamech položek bude zahrnut seznam tabulek. \\

theorems & false & Je-li zapnuto, pak v seznamech bude zahrnut seznam teorémů. \\

sourcecodes & false & Je-li zapnuto, pak v seznamech bude zahrnut seznam zdrojových kódů. \\

glossaries & false & Je-li zapnuto, pak na konci dokumentu bude vysázen seznam zkratek. \\

index & false & Zapíná podporu sazby rejstříku. \\

biblatex & true & Zapne sazbu bibliografie přes balík \BibLaTeX{}.
\end{tabular}}
\end{center}
\end{table}

\subsection{Geometrie stránky}
Tento styl používá list velikosti $A4$. Pro sazbu prací je třeba použít jednostrannou sazbu. Levý okraj je rozšířen s ohledem na vazbu výsledné knižní podoby práce.

\section{Sazba částí dokumentu}
\subsection{Sazba úvodní strany či obsahu}
Vysázení všech podstatných částí úvodu práce obstará makro \kiinlinecode{TeX}{!}{\\maketitle}. Pro správné vysázení všech částí a meta-informací je potřeba použí makra \kiinlinecode{TeX}{!}{\\title}, \kiinlinecode{TeX}{!}{\\author} a další. Jejich přehled lze najít ve zdrojovém souboru tohoto dokumentu. V případě použítí \textbf{pdf} výstupu se generuje i dodatečná hlavička souboru s meta-informacemi jako je autor dokumentu, název práce či dalšími.

\subsection{Závěry}
Závěr práce by se měl poskytnout jak v původním jazyce práce, tak v jazyce anglickém. Pro sazbu závěru jsou k dispozici příslušná makra. Berte na vědomí, že v anglickém závěru se aktivuje plně anglická sazba se všemi konvencemi. Tedy je třeba používat anglické uvozovky a další správné typografické prvky.

\begin{kicode}{TeX}{}{Sazba závěrů}
% Tiskne český závěr práce.
\begin{kiconclusions}
Závěr práce v \uv{českém} jazyce.
\end{kiconclusions}

% Tiskne anglický závěr práce.
\begin{kiconclusions}[english]
Thesis conclusions written in \uv{English}.
\end{kiconclusions}
\end{kicode}

\subsection{Matematika}
Pro sazbu matematiky je k dispozici sada standardních maker.
$$\langle f \rangle, \lfloor g \rfloor,
\lceil h \rceil, \ulcorner i \urcorner$$

$$\left\{\frac{x^2}{y^3}\right\}$$

$$
A_{m,n} =
 \begin{pmatrix}
  a_{1,1} & a_{1,2} & \cdots & a_{1,n} \\
  a_{2,1} & a_{2,2} & \cdots & a_{2,n} \\
  \vdots  & \vdots  & \ddots & \vdots  \\
  a_{m,1} & a_{m,2} & \cdots & a_{m,n}
 \end{pmatrix}
$$

$$
M = \begin{bmatrix}
       \frac{5}{6} & \frac{1}{6} & 0           \\[0.3em]
       \frac{5}{6} & 0           & \frac{1}{6} \\[0.3em]
       0           & \frac{5}{6} & \frac{1}{6}
     \end{bmatrix}
$$

\subsection{Sazba literatury}
Pro sazbu literatury má uživatel dvě možnosti. Může použít služeb balíků \BibLaTeX{}, který je pro \textbf{kistyles} zapnutý, či lze použít manuální sazbu bibliografie.
\subsubsection{Sazba bibliografie přes \BibLaTeX{}}
Při použití tohoto balíku se data o použité literatuře ukládají do dedikovaného textového souboru, ukázku najdete i v tomto stylu pod jménem \kiinlinecode{text}{!}{bibliografie.bib}.

Formát daného souboru je nad rámec této dokumentace a je na každém uživateli, aby si jej nastudoval. Bibliografie se tiskne makrem \kiinlinecode{TeX}{!}{\\printbibliography}. Taktéž v preambuli dokumentu je třeba definovat, který soubor data bibliografie obsahuje, tedy například \kiinlinecode{TeX}{!}{\\bibliography\{bibliografie.bib\}}.

Dokument, který využívá \BibLaTeX{} je následně nutné přeložit jak pomocí překladače zvoleného ovladače, tak pomocí aplikace \kiinlinecode{text}{!}{biber}. Více informací poskytne soubor \kiinlinecode{text}{!}{Makefile} z distribuce tohoto stylu.

Výhodou tohoto přístupu je, že bibliografie se vysází automaticky a (obvykle) není třeba manuální úprava formátování.

\subsubsection{Manuální sazba bibliografie}
Manuální sazba obnáší vysázení prostředí \kiinlinecode{text}{!}{thebibliography} ručně. To je nad rámec tohoto dokumentu. Ukázku tohoto přístupu lze samozřejmě nalézt ve zdrojovém souboru tohoto dokumentu nebo také \href{http://www.math.uiuc.edu/~hildebr/tex/bibliographies.html}{zde}.

Pro aktivaci manuální sazby bibliografie je třeba volat třídu \kiinlinecode{text}{!}{kidiplom} s parametrem \kiinlinecode{text}{!}{biblatex=false}. Mějte, prosím, na paměti, že v tomto módu jsou makra \kiinlinecode{text}{!}{\\bibliography} a \kiinlinecode{text}{!}{\\printbibliography} nedostupná.

\subsection{Drobná makra}
Základní styl definuje hned několik maker pro usnadnění práce. Například makro \kiinlinecode{TeX}{!}{\\buno} vysází řetezec \uv{bez újmy na obecnosti}. Je k dispozici i verze s prvním velkým písmenem, \kiinlinecode{TeX}{!}{\\Buno}.

Je rovněž možno přidávat položky do seznamu zkratek. K tomu slouží makro \kiinlinecode{TeX}{!}{\\newacronym}, které lze použít například jednoduše jako \kiinlinecode{TeX}{!}{\\newacronym\{UPOL\}\{UPOL\}\{\\kitextunivcz\}}. Na danou zkratku se pak lze odkazovat jednoduše, \kiinlinecode{TeX}{!}{\\gls\{UPOL\}}.

Sazba uvozovek respektuje nastavení částí dokumentu, a proto se doporučuje používat makro \kiinlinecode{TeX}{!}{\\uv}. V anglické závěru práce toto platí taky, viz tato PDF ukázka.

Styl podporuje sazbu odstavců v tabulkách, více obsahuje tabulka \ref{tab:odstavce}.

\begin{table}
\begin{center}
\caption{Seznam přepínačů}\label{tab:odstavce}
\begin{tabular}{L{4cm}|R{4cm}|L{4cm}}
\lipsum[23] & \lipsum[22] & \lipsum[21]
\end{tabular}
\end{center}
\end{table}

K dispozici jsou také makra pro sazbu \csharp{} (\kiinlinecode{TeX}{!}{\\csharp}) či \cpp{} (\kiinlinecode{TeX}{!}{\\cpp}).

%% v případě tvorby rejstříku přeložit vygenerovaný soubor .idx
%% programem Makeindex a v případě tvorby seznamu zkratek spustit
%% program Makeglossaries s parametrem jméno souboru zdrojového textu
%% bez přípony a následně opět (dvakrát) přeložit zdrojový text
%% programem pdfLaTeX.

\subsection{Sazba rejstříku}
Sazba rejstříku sestává z několika kroků:

\begin{enumerate}
\item Je třeba přes volbu \kiinlinecode{TeX}{!}{index=true} rejstříkování povolit.
\item Použítím makra \kiinlinecode{TeX}{!}{\\index} rejstříkovat vybrané pojmy.
\item Kompilovat s použitím utility \kiinlinecode{TeX}{!}{makeindex}. Pro specifika tohoto kroku si stačí prohlédnout soubor \kiinlinecode{text}{!}{Makefile}.
\end{enumerate}

Makro \kiinlinecode{TeX}{!}{\\index} je redefinováno tak, že sází klikací odkaz na výraz v rejstříku. Je doporučeno jej použít ihned za výrazem\index{výraz}.

\textbf{Omezení redefinovaného makra \kiinlinecode{TeX}{!}{\\index}}: klikací odkaz nefunguje, pokud použijete konstrukci \kiinlinecode{TeX}{!}{\\index\{výraz|makro\}} (resp. \kiinlinecode{TeX}{!}{\\index\{výraz|(makro\}}), např. \kiinlinecode{TeX}{!}{\\index\{výraz|textit\}}.

Rejstřík lze vysázet pomocí makra \kiinlinecode{TeX}{!}{\\printindex}.

\subsection{Sazba zdrojových kódů}
Styl nabízí dva způsoby sazby zdrojových kódů:

\begin{enumerate}
\item Sazbu řádkových kódů, například \kiinlinecode{CSS}{!}{background-color: white;}. K tomu slouží makro formátu \kiinlinecode{TeX}{!}{\\kiinlinecode\{jazyk\}\{separátor\}\{kód\}}. Za separátor je vhodné volit jakýkoliv znak, který se nevyskytuje v samotném sázeném zdrojovém kódu. Za jazyk je nutno dosadit jeden z těchto: C, TeX, PHP, HTML, Lisp, SQL, TeX, Python, Java, TutorialD, text, csharp, cpp, JavaScript, CSS.

\item Sazbu zdrojových kódu do separátních prostředí. Takto vytištěný kód se objeví v seznamu zdrojových kódů. Ukázka například zdrojový kód \ref{kod:cpp}. Ukázku sazby naleznete ve zdrojovém kódu tohoto dokumentu.
\end{enumerate}

\newacronym{UPOL}{UPOL}{\kitextunivcz}

\begin{definition}[Název definice]
Abcd. Abcd. Abcd. Abcd. Abcd. Abcd. Abcd. Abcd. Abcd. Abcd. Abcd. Abcd. Abcd. Abcd. Abcd. Abcd. Abcd. Abcd. Abcd. Abcd. Abcd. Abcd. Abcd. Abcd. Abcd. Abcd. Abcd. Abcd. Abcd. Abcd. \gls{UPOL}
\end{definition}

\begin{proof}[Název důkazu]
Abcd. Abcd. Abcd. Abcd. Abcd. Abcd. Abcd. Abcd. Abcd. Abcd. Abcd. Abcd. Abcd. Abcd. Abcd. Abcd. Abcd. Abcd. Abcd. Abcd. Abcd. Abcd. Abcd. Abcd. Abcd. Abcd. Abcd. Abcd. Abcd. Abcd. 
\end{proof}

\begin{remark}[Pumpovací věta]
Abcd. Abcd. Abcd. Abcd. Abcd. Abcd. Abcd. Abcd. Abcd. Abcd. Abcd. Abcd. Abcd. Abcd. Abcd. Abcd. Abcd. Abcd. Abcd. Abcd. Abcd. Abcd. Abcd. Abcd. Abcd. Abcd. Abcd. Abcd. Abcd. Abcd. 
\end{remark}

\begin{example}[Pumpovací věta]
Abcd. Abcd. Abcd. Abcd. Abcd. Abcd. Abcd. Abcd. Abcd. Abcd. Abcd. Abcd. Abcd. Abcd. Abcd. Abcd. Abcd. Abcd. Abcd. Abcd. Abcd. Abcd. Abcd. Abcd. Abcd. Abcd. Abcd. Abcd. Abcd. Abcd. 
\end{example}

\begin{lemma}[Název definice]
Abcd. Abcd. Abcd. Abcd. Abcd. Abcd. Abcd. Abcd. Abcd. Abcd. Abcd. Abcd. Abcd. Abcd. Abcd. Abcd. Abcd. Abcd. Abcd. Abcd. Abcd. Abcd. Abcd. Abcd. Abcd. Abcd. Abcd. Abcd. Abcd. Abcd. 
\end{lemma}

\begin{consequence}[Název důkazu]
Abcd. Abcd. Abcd. Abcd. Abcd. Abcd. Abcd. Abcd. Abcd. Abcd. Abcd. Abcd. Abcd. Abcd. Abcd. Abcd. Abcd. Abcd. Abcd. Abcd. Abcd. Abcd. Abcd. Abcd. Abcd. Abcd. Abcd. Abcd. Abcd. 
\end{consequence}

\begin{theorem}[Pumpovací věta]
Abcd. Abcd. Abcd. Abcd. Abcd. Abcd. Abcd. Abcd. Abcd. Abcd. Abcd. Abcd. Abcd. Abcd. Abcd. Abcd. Abcd. Abcd. Abcd. Abcd. Abcd. Abcd. Abcd. Abcd. Abcd. Abcd. Abcd. Abcd. Abcd. Abcd. 
\end{theorem}


\begin{kicode}{cpp}{kod:cpp}{\cpp}
int main("cs acsa") // komentar
int main("cs acsa") // komentar
int main("cs acsa") // komentar
int main("cs acsa") // komentar
int main("cs acsa") // komentar
\end{kicode}

\begin{kicode}{JavaScript}{}{JS}
new object() // komentar
\end{kicode}

\begin{kicode}{csharp}{}{\csharp}
public static int main("cs acsa") // komentar
\end{kicode}

\begin{kicode}{SQL}{}{SQL}
SELECT * FROM table_1; /* komentar */
\end{kicode}

\begin{kicode}{TutorialD}{}{TutorialD}
table_1 AND table_2;
\end{kicode}

%% Závěry práce. V jazyce práce a anglicky. Text pro jiný než
%% nastavený jazyk práce (nepovinným parametrem language makra
%% \documentclass, výchozí český) se zadává použitím makra s uvedením
%% jazyka jako nepovinného parametru.
\begin{kiconclusions}
Závěr práce v \uv{českém} jazyce.
\end{kiconclusions}

\begin{kiconclusions}[english]
Thesis conclusions in \uv{English}.
\end{kiconclusions}

%% Přílohy obsahu textu práce, za makrem \appendix.
\appendix

\section{První příloha}
Text první přílohy

\section{Druhá příloha}
Text druhé přílohy

%% Obsah přiloženého CD/DVD. Poslední příloha. Upravte podle vlastní
%% práce!
\section{Obsah přiloženého CD/DVD} \label{sec:ObsahCD}

Na samotném konci textu práce je uveden stručný popis obsahu
přiloženého CD/DVD, tj.~jeho závazné adresářové struktury, důležitých
souborů apod.

\begin{description}

\item[\texttt{bin/}] \hfill \\
  Instalátor \textsc{Instalator} programu, popř.~program
  \textsc{Program}, spustitelné přímo z~CD/DVD. / Kompletní adresářová
  struktura webové aplikace \textsc{Webovka} (v~ZIP archivu) pro
  zkopírování na webový server. Adresář obsahuje i~všechny runtime
  knihovny a~další soubory potřebné pro bezproblémový běh instalátoru
  a~programu z~CD/DVD / pro bezproblémový provoz webové aplikace na
  webovém serveru.

\item[\texttt{doc/}] \hfill \\
  Text práce ve formátu PDF, vytvořený s~použitím závazného stylu KI
  PřF UP v~Olomouci pro závěrečné práce, včetně všech příloh,
  a~všechny soubory potřebné pro bezproblémové vygenerování PDF
  dokumentu textu (v~ZIP archivu), tj.~zdrojový text textu, vložené
  obrázky, apod.

\item[\texttt{src/}] \hfill \\
  Kompletní zdrojové texty programu \textsc{Program} / webové aplikace
  \textsc{Webovka} se všemi potřebnými (příp.~převzatými) zdrojovými
  texty, knihovnami a~dalšími soubory potřebnými pro bezproblémové
  vytvoření spustitelných verzí programu / adresářové struktury pro
  zkopírování na webový server.

\item[\texttt{readme.txt}] \hfill \\
  Instrukce pro instalaci a~spuštění programu \textsc{Program}, včetně
  všech požadavků pro jeho bezproblémový provoz. / Instrukce pro
  nasazení webové aplikace \textsc{Webovka} na webový server, včetně
  všech požadavků pro její bezproblémový provoz, a~webová adresa, na
  které je aplikace nasazena pro účel testování při tvorbě posudků
  práce a~pro účel obhajoby práce.

\end{description}

Navíc CD/DVD obsahuje:

\begin{description}

\item[\texttt{data/}] \hfill \\
  Ukázková a~testovací data použitá v~práci a~pro potřeby testování
  práce při tvorbě posudků a~obhajoby práce.

\item[\texttt{install/}] \hfill \\
  Instalátory aplikací, runtime knihoven a~jiných souborů potřebných
  pro provoz programu \textsc{Program} / webové aplikace
  \textsc{Webovka}, které nejsou standardní součástí operačního
  systému určeného pro běh programu / provoz webové aplikace.

\item[\texttt{literature/}] \hfill \\
  Vybrané položky bibliografie, příp.~jiná užitečná literatura
  vztahující se k~práci.

\end{description}

U~veškerých cizích převzatých materiálů obsažených na CD/DVD jejich
zahrnutí dovolují podmínky pro jejich šíření nebo přiložený souhlas
držitele copyrightu. Pro všechny použité (a~citované) materiály,
u~kterých toto není splněno a~nejsou tak obsaženy na CD/DVD, je uveden
jejich zdroj (např.~webová adresa) v~bibliografii nebo textu práce
nebo v souboru \texttt{readme.txt}.

%% -------------------------------------------------------------------

%% Sazba volitelného seznamu zkratek, za přílohami.
\printglossary

%% Sazba povinné bibliografie, za přílohami (případně i za seznamem
%% zkratek). Při použití BibLaTeXu použijte makro
%% \printbibliography. jinak prostředí thebibliography. Ne obojí!

%% Sazba i v textu necitovaných zdrojů, při použití
%% BibLaTeXu. Volitelné.
\nocite{*}
%% Vlastní sazba bibliografie při použití BibLaTeXu.
\printbibliography

%% Bibliografie, včetně sazby, při nepoužití BibLaTeXu.
% \begin{thebibliography}{9}
%\bibitem{kniha2} \uppercase{Hawke}, Paul. NanoHttpd: Light-weight HTTP server designed for embedding in other applications. GitHub [online]. 2014-05-12. [cit. 2014-12-06]. Dostupné z: \url{https://github.com/NanoHttpd/nanohttpd}
%
%\bibitem{jeske13} \uppercase{Jeske}, David; \uppercase{Novák}, Josef. Simple HTTP Server in \csharp: Threaded synchronous HTTP Server abstract class, to respond to HTTP requests. CodeProject: For those who code [online]. 2014-05-24. [cit. 2014-12-06]. Dostupné z: \url{http://www.codeproject.com/Articles/137979/Simple-HTTP-Server-in-C}
%
%\bibitem{uzis2012} \uppercase{ÚSTAV ZDRAVOTNICKÝCH INFORMACÍ A STATISTIKY ČR}. Lékaři, zubní lékaři a farmaceuti 2012 [online]. Praha 2, Palackého náměstí 4: Ústav zdravotnických informací a statistiky ČR, 2012 [cit. 2014-12-06]. ISBN 978-80-7472-089-5. Dostupné z: \url{http://www.uzis.cz/publikace/lekari-zubni-lekari-farmaceuti-2012}
% \end{thebibliography}

%% Sazba volitelného rejstříku, za bibliografií.
\printindex

\end{document}
